%!TEX root = report.tex

\chapter*{Abstract}

The performance characteristics of Wi-Fi networks have traditionally been
studied and analysed using analytical models and simulations. Due to the
complexity of wireless communication the existing analytical Wi-Fi network
models rely on certain network constraints and simplifications in order to be
mathematically tractable.

We set out to evaluate the practicality of using Wi-Fi performance models to 
estimate network performance by collecting the necessary parameters directly
from an access point. By extension, we must also collect network metrics,
such as packet payload size and number of nodes, for comparison with the model. We 
explore different venues to collect these parameters and metrics to find out
if it is practical to apply the models in Wi-Fi networks. 

After performing three attempts, we conclude that this is difficult due to
several aspects in the Linux kernel, such as batching optimization patterns,
proprietary kernel modules and firmware blobs.

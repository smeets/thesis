%!TEX root = report.tex

\chapter*{Abstract}

The performance characteristics of Wi-Fi networks have traditionally been
studied and analysed using analytical models and simulations. Due to the
complexity of wireless communication the existing analytical Wi-Fi network
models rely on certain network constraints and simplifications in order to be
mathematically tractable.

Still, analytical Wi-Fi network models have been proven to mimic Wi-Fi network
simulation models closely. There are, however, no empirical tests to confirm
that these findings still hold in practical scenarios.

We set out to evaluate one prominent model, based on the work of Bianchi
\cite{bianchi}, in two scenarios: in the controlled lab and in the wild. The lab
experiments showed that while the model is difficult to adapt to modern Wi-Fi
networks there is a correlation between the model data and lab data.

It turned out that gathering valid lab data is a hard problem without access
to the hardware in question and a large portion of the thesis effort went into
attempting to sample low-level network information. The first attempt was
based on Wireshark, the second on modelling the Linux network system as a
queueing system and third and final attempt using a modified Wi-Fi driver. The
Wireshark-based solution failed due to Linux delivering packets to Wireshark
before they are actually transmitted. The queuing system approach was
discovered to contain timing errors. Our modified Wi-Fi driver shows some
promise, however after extensive tests we could not determine whether the
obtained data was accurate or not.




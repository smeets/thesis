%!TEX root = report.tex

\chapter{Introduction}

This introductory chapter begins by reviewing Wi-Fi usage in our modern society, and use this to motivate an evaluation of a IEEE 802.11 (Wi-Fi) performance model based on a Markov Chain approximation of the Distributed Coordination Function (DCF). 

At the end of this chapter we describe the methodology, organization and problem definition.

\section{Background}

% WiFI nätverk är en viktig del för att ge konnektivitet i hemmiljö, en väldigt
% stor del av hushåll använder en WiFI router för att koppla upp både datorer,
% TV, musikanläggningar och smarta prylar mot nätet. Det har varit känt en
% längre tid att WiFi kan ge en dålig prestanda (QoS) vilket gör användare
% missnöjda. Situationen blir bara värre med ökad strömning av TV och video samt
% videokonferenser vid ökat hemarbete.

Wi-Fi seems to be the wireless home network protocol of the (forseeable)
future. In 2014 a report on Wi-Fi adoption found that 25\% of households, all
over the world, had set up Wi-Fi networks. In households with fixed-line
broadband access, 65\% had set up a Wi-Fi network\cite{smith}. The report also
states that the number of Wi-Fi-enabled devices is projected to increase.

Consumers today have higher expectations regarding network throughput than the
original IEEE 802.11 standard was designed for back in the mid 90's. In recent years,
the Wi-Fi label has become hugely popular and the number of Wi-Fi-capable
devices have skyrocketed, especially in urban areas and neighbourhoods. The
protocol that once was aimed at corporate sector is now almost everywhere
around us, and in vastly different use-cases than for which it first was designed.
Beside home network use, Wi-Fi networks are also deployed for mobile network
off-loading \cite{offloading}.

Alongside this explosion of households relying on a Wi-Fi router to connect
their everyday electronics (computers, phones, TVs) and ``smart devices''
(i.e. internet-connected devices), our usage patterns and quality expectations
have similarly increased: video streaming in 1080p and even UHD (4K) is now
possible on many platforms.

But the increased Wi-Fi usage does not come without problems. It has become
widely known among consumers that Wi-Fi can exhibit poor performance (for a
multitude of reasons). A user streaming video (a use case where user
experience is sensitive to throughput) to their TV will have a significant
impact on the quality of service another user on the same network experiences
on their video conference (sensitive to latency \& jitter). As more people
attempt to work remotely this type of network contention can only be expected
to become more common.

Ignoring the physical aspects involved, the primary task of a wireless network
protocol is to share the underlying medium to all clients in an effective
manner. As with all radio technologies, Wi-Fi is primarily constrained by the
radio spectrum it can utilize. All available performance is derived from a
clever exploitation of this physical medium. As Wi-Fi usage has increased, the
corresponding increase in radio spectrum usage, and the resulting issues of media
sharing and interference, puts the protocol, and its medium access mechanisms
in particular, under even more pressure.

\section{Motivation}

% Vi ställer oss frågan om det går att mäta prestandan i WiFI nätverket och från
% dessa mätningar identifiera om tjänsteleveransen är tillräckligt bra för att
% stödja de tjänster som används för tillfället. Om detta är möjligt kan man i
% förlängningen utveckla algoritmer som WiFI routrar använder för att
% automatiskt konfigurera sig så att prestandan blir tillräcklig.

The performance of a household Wi-Fi network is not solely determined by a router
or the broadband connection. Factors such as network configuration (channel
settings, guard intervals, access modes), environment (noisy neighbours?), and
clients (e.g. hardware and Wi-Fi generation) have a major impact on the ultimate
network performance perceived by users.

To meet a wide variety of customer expectations in scenarios such as VoIP,
low-latency gaming, ultra-high definition streaming and many network nodes,
Wi-Fi has evolved dramatically in complexity, resulting in a multitude of
configurable parameters. Even though newer routers are able to (somewhat)
automatically (re)configure themselves based on analysis of neighbouring
networks, they are not guaranteed to be optimal since they have a local view
of the network (i.e. one point-of-view). Older devices rely on manual
configuration, often using factory defaults.

If possible, could measuring (the right) Wi-Fi performance metrics shed some
light on why the perceived network performance is poor in a given situation?
Could the performance metrics be used to construct expert-type systems? In
addition would these metrics be useful in the development of autonomous
(re)configuration algorithms, embedded in a router? How would these algorithms
be designed? Our position is that a reasonable–for a reasonable definition
of the definition–approach is to build algorithms upon a model.

% @ARTICLE{490421, author={M. M. -. {Cheng} and J. C. -. {Chuang}},
% @journal={IEEE Journal on Selected Areas in Communications},
% @title={Performance evaluation of distributed measurement-based dynamic
% @channel assignment in local wireless communications}, year={1996},
% @volume={14}, number={4}, pages={698-710}, doi={10.1109/49.490421}}


% Vi undersöker hur man skall gå tillväga för att samla in användardata på
% Linuxplattformen då denna utgör basen i de allra flesta WiFI produkter på
% marknaden, och därför är av speciellt intresse. Vi undersöker sedan om den
% insamlade informationen är tillräcklig för att tillämpa på en modell som är
% lovande för att bestämma prestandastatus i nätverket (Felemban). Modellen
% bygger på en del antaganden som gör det osäkert om den är tillämpningsbar på
% en verklig mix av parametrar och vi vill därför i förlängningen undersöka om
% modellen är användbar för detta syfte.

Accurately modelling the Wi-Fi communication and related performance
characteristics is an active field of research and today there are various
proposed models which perform well in simulations
\cite{bianchi}\cite{felemban}. Some of these models are based on the observation that Wi-Fi implements \emph{Carrier-sense multiple
access/Collision avoidance} (CSMA/CA)—"listen before speaking"—in a
\emph{Distributed Coordination Function} (DCF) to reduce the likelihood of
collisions happening in the first place, and what to do when collisions occur.
A branch of these models–of which one we will attempt to evaluate in this report–are built on the approximation of the DCF as a Markov
Chain. Furthermore, the models are often constrainted by assumptions,
neccesary for mathematical tractablility, that cast doubt on the models
ability to reflect and perform in the physical world.

Evaluating if the models, despite their assumptions, are useful for
determining network performance could potentially be of enourmous benefit for
consumers, business and ISPs alike. Imagine each router embedding and
periodically running the model with locally sourced data, automatically
alerting the end-user or ISP of potential performance problems and possible
interventions. Who knows, at some point in the future, devices might even
attempt to cooperatively (and autonoumously!) resolve identified network
problems.

In this report we solely collect metrics from devices running Linux, as it is
the foundation of the majority of all Wi-Fi products on the market today. The
same software that runs on a Wi-Fi router can be run on a laptop and since
the Linux kernel is "open source", we can experiment directly with the
software itself. 

\section{Method, Problem Definition and Organization}

We aim to evaluate how well the model presented in \cite{felemban} perform in the physical world. The methodology for our work will be explained in further
detail in following chapters, and is based on analysing the model, collecting and
comparing empirical data with the model. 

Our problem definitions are formed by taking our overall goal of evaluating the Felemban-Ekici model and breaking it into smaller pieces:

\begin{itemize}

\item \emph{Problem 1} - primary question: is the Felemban-Ekici model from
\cite{felemban} \emph{useful} for determining Wi-Fi network performance?

\item \emph{Problem 2} - definition: what is a reasonable definition of
\emph{useful} in this context?

\end{itemize}

Since we have elected to use an experimental methodology, we must also include definitions related to the collection and evaluation of empirical data.

\begin{itemize}
\item \emph{Problem 3} - analysis: what data should be collected?
\item \emph{Problem 4} - experiment: how should the necessary data be collected?
\item \emph{Problem 5} - evaluate: compare collected data with model and our definition from \emph{Problem 2}
\end{itemize}

The remainder of this thesis is organized as follows. 

Chapter 2 provides a background to, and overview of, related systems, protocols
and hardware. 

Chapter 3 introduces the research field and prior works.

In Chapter 4 we present the methodology and experiments dervied from \emph{Problem 2}, \emph{Problem 3} and \emph{Problem 4}. 

We show collected data for \emph{Problem 3} and \emph{Problem 4} in Chapter 5,
and discuss these results with regards to \emph{Problem 5} and \emph
{Problem 1} in Chapter 6 along with ideas for future work and our closing
thoughts.

Some source material can be found in the Appendix. However, please refer to the repository available online at \url{https://github.com/smeets/thesis} for more content and details.



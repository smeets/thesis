%!TEX root = base.tex

\chapter{Introduction}

Wi-Fi seems to be the wireless home network protocol of the (forseeable)
future. In 2014 a report on Wi-Fi adoption found that 25\% of households, all
over the world, had Wi-Fi networks set up. In households with fixed-line
broadband access, 65\% had set up a Wi-Fi network\cite{smith}. The report also
states that the number of Wi-Fi-enabled devices is projected to increase.

Consumers today have higher expectations regarding network throughput than the
original IEEE 802.11 standard was designed for back in 1997. In recent years,
the Wi-Fi label has become hugely popular and the number of Wi-Fi-capable
devices have skyrocketed, especially in urban areas and neighbourhoods. The
protocol that once was aimed at corporate sector is now almost everywhere
around us, in vastly different use-cases than for which it first was designed.

As Wi-Fi usage has increased, the corresponding increase in radio activity
puts the protocol, and its medium access mechanisms in particular, under ever
more pressure. 

%TODO
something something optimising using congestion window?

\section{Outline} 

This introductory chapter describes the motivations to evaluate IEEE 802.11
(Wi-Fi) performance models based on markov chain approximation of the
distributed coordination function (DCF). A problem definition is also provided
to clearly define the scope of this thesis.

Chapter 2, background, provides an overview of related systems, protocols and
hardware, followed by Chapter 3 which introduces the research field and prior
works.

Chapter 4 discusses the experimental methodology used to achieve the different
goals.	

Afterwards, Chapter 5 presents the results which are further discussed in
Chapter 6.

All source material can be found online at \url{https://github.com/smeets/thesis}.

\section{Motivation} 

To meet a wide variety of customer expectations in scenarios such as VoIP,
low-latency gaming, ultra-high definition streaming and many network nodes,
Wi-Fi has evolved dramatically in complexity, resulting in a multitude of
configurable parameters. Even though newer routers are able to (somewhat)
automatically (re)configure themselves based on analysis of neighbouring
networks, they are not guaranteed to be optimal since they have a local view
of the network (i.e. one point-of-view). Older routers rely on manual
configuration, often factory defaults. 

Accurately modelling the Wi-Fi communication and related performance
characteristics is an active field of research and today there are various
proposed models which perform well in simulations. Wi-Fi implements
\emph{Carrier-sense multiple access/Collision avoidance} (CSMA/CA)—"listen
before speaking"—in a \emph{distributed coordination function} (DCF) to reduce
the likelihood of collisions happening in the first place what to do when
collisions occur.  This thesis focuses on a branch of models presented in
\cite{bianchi} which models the \emph{DCF} using a markov chain-based
approach. Modelling the performance of Wi-Fi networks is beneficial in many
cases, especially (re)configuration—where the ability to estimate impact of
different parameters is crucial. 

With existing, theoretical, Wi-Fi models it is possible to simulate the impact
of configuration settings. However, it is unknown how actual hardware
implementations conform to the standard, on which the theoretical models are
based. Furthermore, it's not pratical to suggest that end-users run
simulations themselves to improve network performance. Knowing if the models
in \cite{felemban} are accurate in production networks could potentially be of
enourmous benefit for end-users and ISPs alike. Imagine each router embedding
and periodically running the model with locally sourced data, automatically
alerting the end-user or ISP of potential performance problems and possible
interventions.

\section{Problem Definition}

This thesis aims to test the foundational assumptions made in
\cite{bianchi}—markov chain approximation of the \emph{DCF},
poisson-distributed packet rates and payload sizes—using an experimental
methodology with data measured from physical devices. The model evaluated is
an improved version presented in \cite{felemban}, not the original from
\cite{bianchi}.

The second part of this thesis evaluates whether in-router telemertry can be
used as input to the evaluated model. [TODO]

The third and final part presents a proof-of-concept implementation of such an
input mapping. [TODO]

%!TEX root = report.tex

\chapter{Introduction}

This introductory chapter describes the motivations to evaluate IEEE 802.11
(Wi-Fi) performance models based on Markov Chain approximation of the
Distributed Coordination Function (DCF). A problem definition is also provided
to clearly define the scope of this thesis.

Chapter 2 provides a background to and overview of related systems, protocols
and hardware. It is followed by Chapter 3 which introduces the research field
and prior works. Chapter 4 discusses the experimental methodology used to
achieve the different goals. Afterwards, Chapter 5 presents the results which
are further discussed in Chapter 6.

All source material available online at \url{https://github.com/smeets/thesis}.

\section{Background}

% WiFI nätverk är en viktig del för att ge konnektivitet i hemmiljö, en väldigt
% stor del av hushåll använder en WiFI router för att koppla upp både datorer,
% TV, musikanläggningar och smarta prylar mot nätet. Det har varit känt en
% längre tid att WiFi kan ge en dålig prestanda (QoS) vilket gör användare
% missnöjda. Situationen blir bara värre med ökad strömning av TV och video samt
% videokonferenser vid ökat hemarbete.

Wi-Fi seems to be the wireless home network protocol of the (forseeable)
future. In 2014 a report on Wi-Fi adoption found that 25\% of households, all
over the world, had Wi-Fi networks set up. In households with fixed-line
broadband access, 65\% had set up a Wi-Fi network\cite{smith}. The report also
states that the number of Wi-Fi-enabled devices is projected to increase.

Consumers today have higher expectations regarding network throughput than the
original IEEE 802.11 standard was designed for back in 1997. In recent years,
the Wi-Fi label has become hugely popular and the number of Wi-Fi-capable
devices have skyrocketed, especially in urban areas and neighbourhoods. The
protocol that once was aimed at corporate sector is now almost everywhere
around us, in vastly different use-cases than for which it first was designed.
Beside home network use, Wi- Fi networks are deployed for mobile network
off-loading \cite{offloading}.

Alongside this explosion of households relying on a Wi-Fi router to connect
their everyday electronics (computers, phones, TVs) and ``smart devices''
(i.e. internet-connected devices), our usage patterns and quality expectations
have similarly increased: video streaming in 1080p and even UHD (4K) is now
possible on many platforms.

It has become widely known among consumers that Wi-Fi can exhibit poor
performance (for a multitude of reasons). A user streaming video (sensitive to
throughput) to their TV will have a significant impact on the quality of
service another user on the same network experiences on their video conference
(sensitive to latency \& jitter). As more people attempt to work remotely this
type of network contention can only be expected to become more common.

Ignoring the physical aspects involved, the primary task of a wireless network
protocol is to share the underlying medium to all clients in an effective
manner. As with all radio technologies, Wi-Fi is primarily constrained by the
radio spectrum it can utilize. All available performance is derived from the
clever exploitation of this physical medium. As Wi-Fi usage has increased, the
corresponding increase in radio activity, and the resulting issues of media
sharing and interference, puts the protocol, and its medium access mechanisms
in particular, under ever more pressure.

\section{Motivation}

% Vi ställer oss frågan om det går att mäta prestandan i WiFI nätverket och från
% dessa mätningar identifiera om tjänsteleveransen är tillräckligt bra för att
% stödja de tjänster som används för tillfället. Om detta är möjligt kan man i
% förlängningen utveckla algoritmer som WiFI routrar använder för att
% automatiskt konfigurera sig så att prestandan blir tillräcklig.

The performance of a household Wi-Fi network is not solely determined a router
or the broadband connection. Factors such as network configuration (channel
settings, guard intervals, access modes), environment (noisy neighbours?), and
clients (e.g. hardware and Wi-Fi generation) have a major impact on the
network performance perceived by users.

To meet a wide variety of customer expectations in scenarios such as VoIP,
low-latency gaming, ultra-high definition streaming and many network nodes,
Wi-Fi has evolved dramatically in complexity, resulting in a multitude of
configurable parameters. Even though newer routers are able to (somewhat)
automatically (re)configure themselves based on analysis of neighbouring
networks, they are not guaranteed to be optimal since they have a local view
of the network (i.e. one point-of-view). Older devices rely on manual
configuration, often factory defaults.

If possible, could measuring (the right) Wi-Fi performance metrics shed some
light on why the perceived network performance is poor? Could the performance
metrics be used to construct expert-type systems? And would these metrics be
useful in the development of autonomous (re)configuration algorithms, embedded
in a router?

% @ARTICLE{490421, author={M. M. -. {Cheng} and J. C. -. {Chuang}},
% @journal={IEEE Journal on Selected Areas in Communications},
% @title={Performance evaluation of distributed measurement-based dynamic
% @channel assignment in local wireless communications}, year={1996},
% @volume={14}, number={4}, pages={698-710}, doi={10.1109/49.490421}}


% Vi undersöker hur man skall gå tillväga för att samla in användardata på
% Linuxplattformen då denna utgör basen i de allra flesta WiFI produkter på
% marknaden, och därför är av speciellt intresse. Vi undersöker sedan om den
% insamlade informationen är tillräcklig för att tillämpa på en modell som är
% lovande för att bestämma prestandastatus i nätverket (Felemban). Modellen
% bygger på en del antaganden som gör det osäkert om den är tillämpningsbar på
% en verklig mix av parametrar och vi vill därför i förlängningen undersöka om
% modellen är användbar för detta syfte.

Accurately modelling the Wi-Fi communication and related performance
characteristics is an active field of research and today there are various
proposed models (both analytical and ) which perform well in simulations
\cite{bianchi}\cite{felemban}. Wi-Fi implements \emph{Carrier-sense multiple
access/Collision avoidance} (CSMA/CA)—"listen before speaking"—in a
\emph{Distributed Coordination Function} (DCF) to reduce the likelihood of
collisions happening in the first place, and what to do when collisions occur.
A branch of models are built on the approximation of the DCF as a Markov
Chain. Furthermore, the models are often constrainted by assumptions,
neccesary for mathematical tractablility, that cast doubt on the models
ability to perform in the physical world.

Evaluating if the models, despite their assumptions, are useful for
determining network performance could potentially be of enourmous benefit for
consumers, business and ISPs alike. Imagine each router embedding and
periodically running the model with locally sourced data, automatically
alerting the end-user or ISP of potential performance problems and possible
interventions. Who knows, at some point in the future, devices might even
attempt to cooperatively (and autonoumously!) resolve identified network
problems.


\section{Method and Problem Definitions}

This thesis aims to evaluate how well the branch of Markov Chain-based models,
originally presented in \cite{bianchi}, perform in the physical world. The
model evaluated is an improved version presented in \cite{felemban}.

This question, as with most deceptively simple-sounding questions, must first
be taken apart:

\begin{itemize}

\item \emph{Problem 1} - primary question: is the Felemban-Ekici model from
\cite{felemban} \emph{useful} for determining Wi-Fi network performance?

\item \emph{Problem 2} - definition: what is a reasonable definition of
\emph{useful} in this context?

\end{itemize}

The problem definition also includes work related the collection, evaluation
and potential usage of data collected from a router itself.

\begin{itemize}
\item \emph{Problem 3} - analysis: what data should be collected?
\item \emph{Problem 4} - experiment: how should the necessary data be collected?
\item \emph{Problem 5} - evaluate: compare collected data with model and our definition from \emph{Problem 2}
\end{itemize}

In Chapter 4 we will present a definiton for \emph{Problem 2} as well as
problem \emph{Problem 3} and \emph{Problem 4}. We show collected data for
\emph{Problem 3} and \emph{Probme 4} in Chapter 5, and discuss these results
with regards to \emph{Problem 5} and \emph{Problem 1} in Chapter 6.


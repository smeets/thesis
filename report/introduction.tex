%!TEX root = base.tex

\chapter{Introduction}

In 2014 a report on Wi-Fi adoption found that 25\% of households, all over the
world, had Wi-Fi networks set up. In households with fixed-line broadband
access, 65\% had set up a Wi-Fi network\cite{smith}. The report also states that
the number of Wi-Fi-enabled devices is projected to increase.

Naturally, consumers today have higher expectations regarding network
throughput than the IEEE 802.11 standard was designed for back in 1997. In
recent years, the Wi-Fi label has become hugely popular and the number of
Wi-Fi-capable devices have skyrocketed, especially in urban areas and
neighbourhoods. The corresponding increase is radio activity puts the Wi-Fi
protocol, and its medium access protocols in particular, under preassure.
Wi-Fi implements \emph{Carrier-sense multiple access} (CSMA)—"listen before
speaking"—in a \emph{distributed coordination function} (DCF) to reduce
probability of collisions happening in the first place and what to do when
collisions occur. 

Accurately modelling the Wi-Fi communication and related performance
characteristics is an active field of research and today there are various
proposed models which perform well in simulations. This thesis focuses on a
branch of models presented in \cite{bianchi} which models the \emph{DCF} using
a markov chain-based approach. Modelling the performance of Wi-Fi networks is
beneficial in many cases, especially (re)configuration—where the ability to
estimate impact of different parameters is crucial. 

Even though newer routers are able to automatically (re)configure themselves
based on analysis of neighbouring networks, they are not guaranteed to be
optimal since they have a local view of the network. Older routers rely on
manual configuration, often factory defaults. With existing Wi-Fi models it is
possible to simulate the impact of configuration settings. However, it is
unknown how actual hardware implementations conform to the standard, on which
the theoretical models are based. Furthermore, it's not pratical to suggest
that end-users run simulations themselves to improve network performance.

This thesis aims to test the foundational assumptions made in
\cite{bianchi}—markov chain, poisson-distributed packet rates and payload
sizes—using an experimental methodology with data measured from physical
devices. The model evaluated is not the original one from \cite{bianchi},
instead an improved version presented in \cite{felemban} will be used.

The second part of this thesis evaluates whether in-router telemertry can be
used as input to the evaluated model.

The third and final part presents a proof-of-concept implementation of such an
input mapping.

% This
% configuration is not strictly an issue with the standard, but rather with how
% the routers are being deployed. One can think of the deployment process as
% being decentralised—each household purchases and configures their own router.


% Centralised configuration is still not on the map for the foreseeable future. Expert systems are already in-place. The task becomes on how to improve existing systems---can we come up with a system which guarantees that a recommended change results in an improvement? This would require extensive simulation of the network and the ability to predict the effect of a change in a configuration parameter.
% em —

% Recommend configuration based on both detailed and wide view of network performance.

% To recommend a change it must lead to improvement in some aspect.

% How to be sure of this? Model the network and simulate what happens when the change is made.

% This requires competent Wi-Fi models.

% Must have clear definition of a competent Wi-Fi model is.

% Wi-Fi models with good performance in simulated tests need not be as good in the real world.

% So we have to test this, and we need the data to do it.

% Telenor now has a good platform for us to get the data.

% But can we trust it? Perhaps the logging is inaccurate?

% So we need to analyse the quality of the data.

%  - is it accurate?
%  - is it precise?

% Experiments will be conducted and results will impact model testing and evaluation.

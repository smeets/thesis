%!TEX root = base.tex

\chapter{Introduction}

Wi-Fi seems to be the wireless home network protocol of the (forseeable)
future. In 2014 a report on Wi-Fi adoption found that 25\% of households, all
over the world, had Wi-Fi networks set up. In households with fixed-line
broadband access, 65\% had set up a Wi-Fi network\cite{smith}. The report also
states that the number of Wi-Fi-enabled devices is projected to increase.

Consumers today have higher expectations regarding network throughput than the
original IEEE 802.11 standard was designed for back in 1997. In recent years,
the Wi-Fi label has become hugely popular and the number of Wi-Fi-capable
devices have skyrocketed, especially in urban areas and neighbourhoods. The
protocol that once was aimed at corporate sector is now almost everywhere
around us, in vastly different use-cases than for which it first was designed.

Alongside the explosive increase in Wi-Fi usage and availability, our usage
patterns and requirements have undergone a similar change. On-demand video and
music streaming, mobile gaming, Voice-over-IP and other new forms of internet
services are all expected to run on this now ubiquitous wireless protocol.
Wi-Fi has not stood still since its deployment, though. The last 20 years have seen
extension after extension to expand, improve and adapt the protocol to the
ever changing requirements of the future.

Ignoring the field of beamforming and related subjects, the primary task of a
wireless network protocol is to share the underlying medium to all clients in
an effective manner. As with all radio technologies, Wi-Fi is primarily
constrained by the radio spectrum it can utilize. All available performance is
derived from the clever exploitation of this physical medium. As Wi-Fi usage
has increased, the corresponding increase in radio activity puts the protocol,
and its medium access mechanisms in particular, under ever more pressure.

\section{Outline}

This introductory chapter describes the motivations to evaluate IEEE 802.11
(Wi-Fi) performance models based on markov chain approximation of the
distributed coordination function (DCF). A problem definition is also provided
to clearly define the scope of this thesis.

Chapter 2 provides a background to and overview of related systems, protocols
and hardware. It is followed by Chapter 3 which introduces the research field
and prior works. Chapter 4 discusses the experimental methodology used to
achieve the different goals. Afterwards, Chapter 5 presents the results which
are further discussed in Chapter 6.

All source material can be found online at \url{https://github.com/smeets/thesis}.

\section{Motivation}

To meet a wide variety of customer expectations in scenarios such as VoIP,
low-latency gaming, ultra-high definition streaming and many network nodes,
Wi-Fi has evolved dramatically in complexity, resulting in a multitude of
configurable parameters. Even though newer routers are able to (somewhat)
automatically (re)configure themselves based on analysis of neighbouring
networks, they are not guaranteed to be optimal since they have a local view
of the network (i.e. one point-of-view). Older routers rely on manual
configuration, often factory defaults.

Accurately modelling the Wi-Fi communication and related performance
characteristics is an active field of research and today there are various
proposed models which perform well in simulations. Wi-Fi implements
\emph{Carrier-sense multiple access/Collision avoidance} (CSMA/CA)—"listen
before speaking"—in a \emph{distributed coordination function} (DCF) to reduce
the likelihood of collisions happening in the first place what to do when
collisions occur.  This thesis focuses on a branch of models presented in
\cite{bianchi} which models the \emph{DCF} using a markov chain-based
approach. Modelling the performance of Wi-Fi networks is beneficial in many
cases, especially (re)configuration—where the ability to estimate impact of
different parameters is crucial.

With existing, theoretical, Wi-Fi models it is possible to simulate the impact
of configuration settings. However, it is unknown how actual hardware
implementations conform to the standard, on which the theoretical models are
based. Furthermore, it's not pratical to suggest that end-users run
simulations themselves to improve network performance. Knowing if the models
in \cite{felemban} are accurate in production networks could potentially be of
enourmous benefit for end-users and ISPs alike. Imagine each router embedding
and periodically running the model with locally sourced data, automatically
alerting the end-user or ISP of potential performance problems and possible
interventions. Who knows, at some point in the future, devices might even
attempt to resolve identified network problems themselves.

\section{Problem Definition}

This thesis aims to test the foundational assumptions made in
\cite{bianchi}—markov chain approximation of the \emph{DCF},
poisson-distributed packet rates and payload sizes—using an experimental
methodology with data measured from physical devices. The model evaluated is
an improved version presented in \cite{felemban}.

This question, as with most deceptively simple-sounding questions, must first
be taken apart. After resolving each part individually, the answer to the
original question can then be constructed by joined together all parts.
Derivation and integration.

\begin{itemize}
\item \emph{Problem 1} - is the Felemban model from \cite{felemban} \emph{useful}?
\item \emph{Problem 2} - what is a useful definition of \emph{useful}?
\item \emph{Problem 3} - how should the definition of \emph{usefulness} be evaluated?
\end{itemize}

The problem definition also included work related the collection, evaluation
and potential usage of data collected from a router itself.

\begin{itemize}
\item \emph{Problem 4} - what data should be collected?
\item \emph{Problem 5} - how should this be done?
\item \emph{Problem 6} - is this data \emph{useful}?
\end{itemize}

Finally, a proof-of-concept of such a system was to be implemented and evaluated.

\begin{itemize}
\item \emph{Problem 7} - use the collected data, or derivatives of it and run the
model with this data as input
\item \emph{Problem 8} - how does the observed network state (via collected router
metrics) and the projected network state (from the model) compare?
\end{itemize}

As with most master thesis projects, many of these questions are answered
beforehand by others. In the case of this particular thesis, these answers set
us on a path that we had not expected. After reading this thesis it should
hopefully become clear why some of these problems were left out and thus
remain unanswered.

\documentclass[11pt,twoside]{eitExjobb}
%%\documentclass[11pt,twoside,final]{eitExjobb}  % Use final for the final version that will be printed
%%%%%%%%%%%%%%%%%%%%%%%%%%%%%%%%%%%%%%%%%%%%%%%%%%%%
%% Other fonts (Palatino as rm font, helvetica as sf font and courier as tt font. All fonts are normally installed with a standard LaTeX distribution.)
% \usepackage{mathpazo} % Also in math mode
% \usepackage[scaled=.95]{helvet}
% \usepackage{courier}
%%%%%%%%%%%%%%%%%%%%%%%%%%%%%
% ÅÄÖ
\usepackage[utf8]{inputenc}  % Input encoding (this file): 8 bit unicode. Default by most text editors
\usepackage[T1]{fontenc}     % Output encoding (pdf file)
%%%%%%%%%%%%%%%%%%%%%%%%%%%%%
%% Packages used in the example
\usepackage{graphicx}   % Included graphics and some resizable boxes
\usepackage{url}        % nice urls with line breaks
\usepackage{lipsum}     % nonsense text blocks
%%%%%%%%%%%%%%%%%%%%%%%%%%%%%%
%%%%%%%%%%%%%%%%%%%%%%%%%%%%%%
%%%%%%%%%%%%%%%%%%%%%%%%%%%%%%
\begin{document}
%%% Title page
\Title{A Test of \texttt{eitExjobb.cls}}
\Author{Stefan Höst\\\texttt{stefan.host@eit.lth.se}}
\Supervisor{Stefan Höst}
\Examiner{Stefan Höst}
%% Set by default
% \Date{\today}    %% Today's date
% \Year{\the\year} %% This year shown in copyright
% \Company{Department of Electrical and Information Technology\\Lund University}
%%%%%%
\MakeTitlePage  %%% Print title page and copyright page
%%%%%% 
\frontmatter    %%% Page numbering for front pages (small roman)
%%%%% Abstract
\chapter*{Abstract}
\lipsum[1-2]
%%%%% Abstract
\chapter*{Popular Science Summary}
\lipsum[3-7]
%% ToC etc
\tableofcontents
\listoffigures
\listoftables
\cleardoublepage
%%%%% Page numbering for the main thesis (arabic)
\mainmatter
%%%%%

\chapter{Introduction}
\lipsum[1]

\section{Test section}
\lipsum[2-4]

\subsection{Test subsection}
\lipsum[5]

\subsection{New subsection}
\lipsum[6]

\section{New section}
\lipsum[1]
\begin{displaymath}
  A=\sum_{i=0}^{\infty}\beta\alpha^i=\frac{\beta}{1-\alpha},\quad |\alpha|<1
\end{displaymath}
\lipsum[2]

\section{Another section}
\lipsum[3-6]

\section{Yet another section}
\lipsum[7]

%%%%%%%%%%%%%%%%%%%
\chapter[A Shorter Chapter Title]{New Chapter with a Long Title that Spanns over More Than One Line}
In Figure~\ref{fig:testfig} a typical test mage is shown.
\begin{figure}[htbp]
  \centering
  \includegraphics[width=0.4\linewidth]{example-image}
  \caption{Example image.}
  \label{fig:testfig}
\end{figure}

\section{New new section}
\lipsum[1-2]
\begin{table}[htbp]
  \centering
  \begin{tabular}{lll}
    Group & Test 1 & Test 2\\\hline
    A & 253 &54\\
    B & 636 & 33
  \end{tabular}
  \caption{A nice table.}
  \label{tab:tabletest}
\end{table}

\lipsum[3]
\begin{figure}[htbp]
  \begin{minipage}[t]{0.5\linewidth}
    \centering
    \includegraphics[width=0.8\linewidth]{example-image-a}
    \caption{Image A}
    \label{fig:imageA}
  \end{minipage}%
  \begin{minipage}[t]{0.5\linewidth}
    \centering
    \includegraphics[width=0.8\linewidth]{example-image-b}
    \caption{Image B. It can also be a long caption even if the space is narrow.}
    \label{fig:imageB}
  \end{minipage}
\end{figure}

Figure~\ref{fig:imageA} is displayed next to Figure~\ref{fig:imageB}. Notice that \verb|\linewidth| is the line width inside the \verb|minipage|.

\lipsum[4]

\chapter{Comments on LaTeX references}
If you want to know more about \LaTeX\ there is a (free) manual at \cite{cite:NotShort}. For more specific questions, it is recommended to have a look at the forum StackExchange \cite{cite:TeX.SX}, where the most common questions already have answers. All official packets can be found, and downloaded, from CTAN \cite{cite:CTAN}. Finally, for the hardcore programmer who thinks \LaTeX\ is a bit inflexible, I can recommend the \TeX\ introduction in \cite{cite:TeXimpatient}.

For questions about how you should do to get your imported graphics as you want, have a look at \cite{cite:ImportedGraphics}. If you instead want to do the images inline from the \TeX\ code you are recommended to use TikZ \cite{cite:TikZ}. However, it is known to have a relatively high learning threshold. 

%%%%%%%%%%%%%%%%%%%%%%%%%%%%%%%%%%%%%%%
%% References
\begin{thebibliography}{99}
\bibitem{cite:NotShort} T. Oetiker, H Partl, I Hyna, and E. Schlegl, \textit{A (Not So) Short Introduction to \LaTeX2e}, \url{www.ctan.org/tex-archive/info/lshort/english/}
\bibitem{cite:TeX.SX} \{\TeX\} StackExchange, \url{tex.stackexchange.com/}
\bibitem{cite:CTAN} The Comprehensive \TeX\ Archive Network, \url{www.ctan.org/}
\bibitem{cite:TeXimpatient} P. Abrahams, K. Hargreaves, and K. Berry, \textit{TeX for the Impatient}, \url{savannah.gnu.org/projects/teximpatient/}
\bibitem{cite:ImportedGraphics} K. Reckdahl, \textit{Using Imported Graphics in LaTeX and pdfLaTeX}, \url{www.ctan.org/tex-archive/info/epslatex/english}
\bibitem{cite:TikZ} T. Tantau, \textit{The TikZ and PGF Packages}, \url{www.bu.edu/math/files/2013/08/tikzpgfmanual.pdf}. Warning: 400+ pages.
\end{thebibliography}


%%%%%%%%%%%%%%%%%%
\appendix
%%%%%%%%%%%%%%%%%%
\chapter{Some extra material}
\lipsum[1]
\begin{figure}[htbp]
  \centering
  \includegraphics[width=0.6\linewidth]{example-image}
  \caption{A picture or table in the appendix is numbered accordingly.}
  \label{fig:AppFig}
\end{figure}

\end{document}

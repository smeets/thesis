\chapter{Background}

This chapter provides an overview of the related topics for this master thesis.

\section{TG799-vac}

The OpenWrt-based router examined in this thesis, see Figure \ref{fig:tg799}, is commonly known as TG799, manufactured by Technicolor with Broadcom and Quantenna modems. A custom firmware was used to gain root access over SSH.

\begin{figure}
\center
\includegraphics[width=0.5\textwidth]{images/tg799.png}
\caption{The TG799 router from Technicolor}
\label{fig:tg799}
\end{figure}

\section{IEEE 802.11}

The ubiquitous set of LAN standards for wireless communication. Specify MAC and PHY protocols.

Basic protocol description, relevant for this master thesis.

How later standards IEEE 802.11b g n ac change.

\section{ubus - the OpenWrt micro bus architecture}

A client program which acts as an interface to the bus daemon, \texttt{ubusd}. Input and output format is JSON.

The router registers a namespace called \texttt{wireless} and this is what we call to find the meaning of life.

\section{Deutche telekom supra}

Björns oklara vapen.

\section{Rhode \& Schwarz FWLZ-XYZ}

(not broken) network analyser;

\section{Wi-Spy Channalyzer}

the wi-spy, top secret agent auf d00m.

\section{Wireshark}

Wireshark is a well-known program for capturing and inspecting network data.

\section{jana}

A program, developed by the author of this thesis, for running network tests. Supports expontential, uniform and gamma distributed packet send rate and payload size. It is designed to be used together with packet capture software (e.g. Wireshark) which enables a user to estimate the time from a `sendmsg` syscall to the packet actually leaving the NIC. These features were developed to facilitate experimental, yet controlled, evaluation of the theoretical IEEE 802.11 performance models.
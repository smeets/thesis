In Figure~\ref{fig:testfig} a typical test mage is shown.
\begin{figure}[htbp]
  \centering
  \includegraphics[width=0.4\linewidth]{example-image}
  \caption{Example image.}
  \label{fig:testfig}
\end{figure}

\section{New new section}
\lipsum[1-2]
\begin{table}[htbp]
  \centering
  \begin{tabular}{lll}
    Group & Test 1 & Test 2\\\hline
    A & 253 &54\\
    B & 636 & 33
  \end{tabular}
  \caption{A nice table.}
  \label{tab:tabletest}
\end{table}

\lipsum[3]
\begin{figure}[htbp]
  \begin{minipage}[t]{0.5\linewidth}
    \centering
    \includegraphics[width=0.8\linewidth]{example-image-a}
    \caption{Image A}
    \label{fig:imageA}
  \end{minipage}%
  \begin{minipage}[t]{0.5\linewidth}
    \centering
    \includegraphics[width=0.8\linewidth]{example-image-b}
    \caption{Image B. It can also be a long caption even if the space is narrow.}
    \label{fig:imageB}
  \end{minipage}
\end{figure}

Figure~\ref{fig:imageA} is displayed next to Figure~\ref{fig:imageB}. Notice that \verb|\linewidth| is the line width inside the \verb|minipage|.

\lipsum[4]

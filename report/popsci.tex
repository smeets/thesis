\chapter*{Popular Science Summary}

Trådlös kommunikation har växt explosionsartat de senaste årtionden och
konsumenters krav och förväntningar på bandbredd ökar allt mer. I många hem
finns trådlösa accesspunkter med stöd för Wi-Fi. De kommer i många skepnader
- inbakade i en router, som komplement vid sidan av eller i form av s.k.
"Repeaters".

Majoriteten av accesspunkter och apparater implementerar Wi-Fi b/g/n vilka kommunicerar via
radio över frekvensband vid 2.4 GHz. Många modernare accesspunkter lägger även
till stöd för Wi-Fi ac med frekvensband vid 5 GHz.

Radiovågors räckvidd är direkt beroende på våglängden och skillnaden i räckvidd
mellan 2.4 GHz och 5 GHz vara mycket stor. Detta medför att accesspunkter i
ett hus kan nå in i grannhuset eller lägenheterna runt omkring, och därmed störa
den trådlösa kommunikationen där, och vice versa.

För att minimera effekten av denna interferens utnyttjar inte varje accesspunkt
hela frekvensbandet som standarden tillåter. Istället ska accesspunkterna
konfigureras så att närliggande accesspunkter inte använder samma del av
frekvensbandet.

Det inte finns någon central myndighet, samordning bland tjänsteleverantörer
eller grannar emellan som genomför denna konfiguration och det är alltså upp
till den enskilde.

En stor del av hur en användare upplever nätverket kan därför direkt påverkas av
hur deras accesspunkt är konfigurerad. Är det många grannar som pratar på samma
kanaler leder detta till nedsatt prestanda, vilket kan visa sig på många olika
sätt.

När kunden inte är nöjd ringer de tjänsteleverantörens support och försöker få
hjälp. Detta kostar företagen miljonbelopp varje år och oftast måste en tekniker
skickas ut för att försöka lösa problemet, som egentligen inte ligger på
leverantörens del av nätet.

Denna studie har använd data från Telenor med existerande modeller av Wi-Fi-nät
för att undersöka deras uppförande och se om de överhuvudtaget går att använda.

Studien har även tagit ett steg till och undersökt om och hur dessa modeller kan
användas för att estimera Wi-Fi-nätets prestanda vid vissa typer av
förändringar, t.ex. antalet grannstationer.

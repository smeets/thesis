%!TEX root = report.tex

\chapter*{Popular Science Summary}

% Obtaining Network Performance for Wi-Fi Model Evaluation
% Insamling av nätverksdata för praktisk undersökning av Wi-Fi-modellering.

Kan Wi-Fi fungera utan krångel? Vi har försökt att samla in datapunkter från
accesspunkter och andra prylar för att se om det går att tillämpa akademiska
modeller av Wi-Fi i syfte att förutsäga, varna och åtgärda prestandaproblem. 

Vi konstaterar att det är mycket krångligt att samla in den data som behövs
för att utvärdera modellernas tillförlitlighet. På grund av vissa designval i
Linuxkärnan är det mycket svårt att utläsa denna data direkt. Drivrutinerna
för nätverkskorten har förmodligen tillgång till den tidsdata vi behöver, men 
vi tror det krävs stor yttre påverkan innan chiptillverkarna aktivt gör den här
typen av data tillgänglig. Förhoppningsvis kan detta bli en funktionalitet i
en framtida generation av Wi-Fi.

Då Wi-Fi bygger på radiokommunikation finns det alltid en risk för störningar.
Modellerna i sig utgår från den koordineringsprocess som alla noder i ett
Wi-Fi nät måste underkasta sig. Processen kallas för "Carrier-sense/Multiple
Access", och beskriver nästan sig själv: varje nod lyssnar på nätets
radiokanal och kontrollerar att ingen annan sänder, innan den försöker skicka
ett meddelande. Modellering av denna process blir snabbt väldigt komplex och
för att göra modellerna matematiskt hanterbara har man gjort ett flertal
antaganden, vilket såklart påverkar modellerns tillförlitlighet i verkliga
Wi-Fi nät.

För att undersöka modellernas tillförlitlighet har vi därför försökt samla in
data från accesspunkter och laptops som ska motsvara de parametrar som
modellerna själva beskriver: packetlast och hur lång tid det tar att få
tillgång till radiokanalen. Den första är trivial att samla in med t.ex.
Wireshark, medans den sistnämnda datapunkten visade sig vara betydligt
besvärligare.

I tre experiment har vi försökt, med blandade resultat. Vårt första
experiment, med det välkända programmet Wireshark, kan enbart samla in
tidsdata om det finns speciellt hårdvarustöd för tidsstämplar. I det andra
experimentet beskrev vi Linuxkärnans nätverkssystem som ett kösystem. På
grund av optimeringar i hur Linuxkärnarn hanterar avbrott så har mätdata från
detta experiment en för stor varians för våra ändamål. I det sista
experimentet grävde vi djupare, in i nätverkskortets drivrutin där vi hittade
några intressanta parametrar. Efter flera tester kan vi dock inte dra några
slutsatser om dessa parametrar faktiskt är vad vi tror de är - vilket gör dem
värdelösa för vårt ändamål.


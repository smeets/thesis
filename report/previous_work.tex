\chapter{Previous Work}

Modelling of Wi-Fi performance is an active field of research. This chapter introduces the models which are evaluated.

% TODO: add specific models
All evaluated models of this master thesis can trace their origins to the seminal work of Giuseppe Bianchi, who, in 2000, publicised a novel approach to Wi-Fi modeling, based on Markov chains\cite{bianchi}, for fully-connected (no hidden nodes), single-hop networks.

\section{Limitations Of The Bianchi Model}
The Bianchi model in \cite{bianchi} puts forth some serious conditions for its tractability. Some conditions arise from the use of a Markov chain itself, and some to further simplify calculations and behavior. One such intrinsic condition imposed by the Markov chain is that the network must be in a steady state, and all transient behavior is discarded.

\subsection{The Saturation Condition}

One of the most severe conditions of the Bianchi model states that the network must be \emph{saturated}, defined as "every node it the network always has something in its send buffer".

\subsection{The Tagged Node}

Another constraint of the Bianchi approach to modelling is the limitation to the \emph{Tagged Node} (TN). In a network with $N$ nodes, the model is built around the assumption that all $N$ nodes behave similarly, and thus argues that modelling one node, named the \emph{Tagged Node}, is good enough to model the network.

\section{Extending The Bianchi Model}

In 2011, Felemban \& Ekici publicised an extended version of Bianchi's model.

\subsection{The Unsaturated Model}

Another contribution made by Felemban \& Ekici lifted the saturation condition on their original model.
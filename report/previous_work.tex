\chapter{Previous Work}

This chapter provides a brief introduction to some important academic efforts this master thesis is based on. In particular, the Bianchi and Felemban models.

\section{Modelling IEEE 802.11}
Modelling of Wi-Fi performance is an active field of research. This chapter introduces the models which are evaluated.

\section{The Bianchi Model}

All evaluated models of this master thesis can trace their origins to the seminal work of Giuseppe Bianchi, who, in 2000, invented a novel approach to Wi-Fi modeling, based on Markov chains\cite{bianchi}, for fully-connected (no hidden nodes), single-hop networks.

The modelputs forth some serious conditions for its tractability. Some conditions arise from the use of a Markov chain itself, and some to further simplify calculations and behavior. One such intrinsic condition imposed by the Markov chain is that the network must be in a steady state, and all transient behavior is discarded.

In short, the key contribution was to model the backoff behaviour, i.e. the distributed co-ordination function ($DCF$), of the protocol using a Markov chain.

\subsection{Limitations Of The Bianchi Model}
The Bianchi model in \cite{bianchi} puts forth some serious conditions for its tractability. Some conditions arise from the use of a Markov chain itself, and some to further simplify calculations and behavior. One such intrinsic condition imposed by the Markov chain is that the network must be in a steady state, and all transient behavior is discarded.

One of the most severe conditions of the Bianchi model states that the network must be \emph{saturated}, defined as "every node it the network always has something in its send buffer".

Another constraint of the Bianchi approach to modelling is the limitation to the \emph{Tagged Node} (TN). In a network with $N$ nodes, the model is built around the assumption that all $N$ nodes behave similarly, and thus argues that modelling one node, named the \emph{Tagged Node}, is good enough to model the network.

\section{Extending The Bianchi Model}

In 2011, Felemban \& Ekici published an extended version of Bianchi's model, where they significantly improved the model's accuracy by introducing a more accurate behaviour of the entry into backoff and the backoff countdown procedure.

In short, when a packet collision is detected the sender goes into a backoff and counts down until it may attempt to retransmit the packet. In \cite{felemban}, the authors introduce a Markov chain to compute the parameters of $P_d$ and $P_f$, probability of countdown and counter freeze, respectively.

\subsection{The Unsaturated Model}

Another contribution made by Felemban \& Ekici lifted the saturation condition on their original model.
